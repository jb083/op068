
\chapter{作品の構造}

\section{概観}

\section{第1楽章}

\musicbegin
	\startbarno=42
	\systemnumbers
	\def\writebarno{\llap{\the\barno\barnoadd}}%
	\def\raisebarno{2\internote}%
	\def\shiftbarno{1.3\Interligne}%
	\def\nbinstruments{1}%   % パート数 2
	\setstaffs{1}{2}%        % 下から1番目は2段
	\setclef{1}{6000}%       % 下から1番目はへ音記号
	\generalsignature{-3}%    % 調号は正の値のときシャープの数
	\generalmeter{\meterfrac{6}{8}}%  % 拍子は8分の6拍子
	\startextract%
		%(1)
		\notes\lpz{C}\cu{C}\ds\ds|\ds\ds\ibsluru{1}{e}\cu{e}\enotes
		\Notes\isluru{0}{c}\qlp{c}|\qu{g}\enotes
		\notes|\tsslur{1}{g}\cl{l}\enotes
		\bar
		%(2)
		\NOTes\tslur{0}{c}\qlp{c}|\isluru{1}{n}\qlp{n}\enotes
		\Notes\sh{c}\qlp{c}|\tslur{1}{n}\ql{n}\enotes
		\Notes|\na{l}\isluru{1}{l}\cl{l}\enotes
		\bar
		%(3)
		\Notes\itieu{0}{d}\qlp{d}|\tslur{1}{n}\ql{n}\ds\enotes
		\Notes\ibl{0}{d}{-3}\ttie{0}\qbp{0}{d}\nbbl{0}\na{c}\isluru{0}{c}\qb{0}{c}\na{b}\qb{0}{b}\tbl{0}\na{a}\qb{0}{a}|\isluru{1}{o}\usf{o}\qlp{o}\enotes
		\bar
		%(4)
		\notes\na{b}\tslur{0}{b}\upz{b}\cl{b}\ds|\tslur{1}{o}\ql{o}\enotes
		\notes\ds|\upz{m}\cl{m}\enotes
		\Notes\qp|\upz{k}\ql{k}\enotes
		\notes\lpz{J}\cu{J}|\upz{j}\cl{j}\enotes
		\bar
		%(5)
		\notes\lpz{G}\cu{G}\ds|\na{i}\upz{i}\cl{i}\ds\enotes
	\zendextract % 頭にzをつけると最後に小節線を表示しない
\musicend{1-42}{第1楽章第42小節から}



\section{第2楽章}

\section{第3楽章: Un poco Allegretto e grazioso}

ブラームスはこの大規模な交響曲の中で, 164小節という小振りな「間奏曲」を用意した.
ベートーヴェン風のスケルツォではなく, より古風なメヌエットのような音楽をここに置いたことは,
ベートーヴェンの交響曲 (例えば第5番) から意識的に距離を置いていることの現れであろう.
しかも, この楽章は全体を通して二拍子で書かれており, 純然たるメヌエットでさえない.
この楽章は完全にブラームス風の音楽であり, この事実ひとつ取ってもブラームスの第1番が「ベートーヴェンの第10番」という評価では言い尽くせないことがよく表れている.

\begin{table}[htbp]
	\centering
	\begin{tabular}{cccc}
		主部 (A) & 中間部 (B) & 再現部 (A') & コーダ \\ \hline
		1--70 & 71--114 & 115--153 & 154--164 \\
		As-Dur, 2/4 & H-Dur, 6/8 & As-Dur, 2/4 & As-Dur, 2/4 (6/8)
	\end{tabular}
	\caption{第3楽章の構成}
	\label{structure of mov3}
\end{table}
構成は比較的単純な三部形式 (A-B-A') だが, 後で見るように再現部A'は主部Aの単調な繰り返しとなることが避けられており,
三部形式の短い楽章にしては変化に富んだ印象を与える.

\musicbegin
	\def\nbinstruments{1}%   % パート数 2
	\setstaffs{1}{2}%        % 下から1番目は2段
	\setclef{1}{6000}%       % 下から1番目はへ音記号
	\generalsignature{-4}%    % 調号は正の値のときシャープの数
	\generalmeter{\meterfrac{2}{4}}%  % 拍子は8分の6拍子
	\startextract%
		%(1)
		\Notes\cmidstaff{\p}\zqu{c}\ibl{0}{H}{0}\qb{0}{HG}|
			\zcharnote{y}{\hspace*{-8truemm}Un poco Allegretto e grazioso}
			\itied{0}{h}\zhl{h}\ibsluru{1}{l}\qu{l}\enotes
		\Notes\zqu{d}\qb{0}{F}\tbl{0}\qb{0}{H}|
			\ibu{1}{k}{-2}\qb{1}{k}\tbsluru{1}{j}\tbu{1}\qb{1}{j}\enotes
		\bar
		%(2)
		\NOtes\ibu{0}{H}{0}\qb{0}{EH}\qb{0}{D}\tbu{0}\qb{0}{H}|
			\zql{e}\ttie{0}\itied{1}{h}\zhl{h}\ibsluru{2}{k}\ibu{1}{k}{-1}\qb{1}{kj}\zql{f}\qb{1}{i}\tbu{1}\tbsluru{2}{j}\qb{1}{j}\enotes
		\bar
		%(3)
		\NOtes\ibu{0}{H}{0}\qb{0}{EH}\qb{0}{D}\tbu{0}\qb{0}{H}|
			\zql{e}\ttie{1}\itied{0}{h}\zhl{h}\ibsluru{2}{k}\ibu{1}{k}{-1}\qb{1}{kj}\zql{f}\qb{1}{i}\tbu{1}\tbsluru{2}{j}\qb{1}{j}\enotes
		\bar
		%(4)
		\Notes\ibu{0}{F}{+1}\qb{0}{C}\tbl{0}\qb{0}{H}|
			\ttie{0}\lq{h}\zhl{e}\ibsluru{1}{i}\ibu{1}{i}{-2}\qb{1}{i}\tbu{1}\tbsluru{1}{h}\qb{1}{h}\enotes
		\Notes\qb{0}{E}\tbu{0}\qb{0}{G}|
			\ibslurd{3}{g}\itied{2}{g}\zql{g}\itieu{0}{i}\qu{i}\enotes
		\bar
		%(5)
		\Notes\ibl{0}{I}{+1}\qb{0}{I}|
			\ttie{0}\zhu{i}\ibl{1}{g}{0}\ttie{2}\qb{1}{g}\enotes
		\Notes\tbl{0}\na{K}\qb{0}{K}|
			\qb{1}{f}\enotes
		\Notes\qb{0}{L}|
			\qb{1}{e}\enotes
		\Notes\tbl{0}\fl{K}\qb{0}{K}|
			\tbslurd{3}{g}\tbl{1}\qb{1}{g}\enotes
	\endextract % 頭にzをつけると最後に小節線を表示しない
\musicend{3-1}{第3楽章冒頭}

第3楽章冒頭を譜例\ref{3-1}に示す. クラリネットで提示される優雅な旋律だが, ブラームスらしく$5$小節を単位とする変則的な構造を取る.
しかも, $2$拍子が$5$小節続くのではなく, $2 + 2 + 3 + 3$という変拍子である.


\begin{music}
	\setlength{\textwidth}{45truemm}
	\startbarno=45
	\systemnumbers
	\def\writebarno{\llap{\the\barno\barnoadd}}%
	\def\raisebarno{2\internote}%
	\def\shiftbarno{1.3\Interligne}%
	\def\nbinstruments{1}%   % パート数 2
	\setstaffs{1}{1}%        % 下から1番目は2段
	\setclef{1}{0000}%       % 下から1番目はへ音記号
	\generalsignature{-4}%    % 調号は正の値のときシャープの数
	\generalmeter{\meterfrac{2}{4}}%  % 拍子は8分の6拍子
	\startextract%
	% \startpiece
		%(1)
		\Notes\uptext{\boxit{B}}\qp\ds\isluru{1}{j}\cl{j}\enotes
		\bar
		%(2)
		\Notes\ibu{0}{i}{-2}\na{i}\qb{0}{i}\qb{0}{h}\qb{0}{g}\tbu{0}\curve 522\tbsluru{1}{l}\qb{0}{f}\enotes
		\bar
		%(3)
		\NOtes\ibl{0}{i}{+2}\na{i}\isluru{1}{i}\qbp{0}{i}\enotes
		\notes\tbbl{0}\tbl{0}\tslur{1}{k}\qb{0}{k}\enotes
		\notes\ibbl{0}{j}{+2}\isluru{2}{j}\qb{0}{jon}\tbbl{0}\tbl{0}\curve512\tslur{2}{m}\isluru{1}{m}\qb{0}{m}\enotes
		\bar
		%(4)
		\Notes\ibl{0}{l}{-2}\na{l}\qb{0}{l}\qb{0}{k}\qb{0}{j}\tbl{0}\tbsluru{1}{i}\fl{i}\qb{0}{i}\enotes
		\bar
		%(5)
		\NOtes\ibl{0}{l}{+2}\fl{l}\isluru{1}{l}\qbp{0}{l}\enotes
		\notes\tbbl{0}\tbl{0}\tslur{1}{n}\fl{n}\qb{0}{n}\enotes
		\notes\ibbl{0}{j}{+2}\isluru{2}{m}\qb{0}{mrq}\tbbl{0}\tbl{0}\curve511\tslur{2}{p}\qb{0}{p}\enotes
		\bar
		% \alaligne
		%(6)
		\NOtes\zq{m}\ql{o}\enotes
		\Notes\ibl{0}{l}{-2}\na{l}\zq{l}\isluru{1}{n}\qb{0}{n}\tbl{0}\na{k}\zq{k}\tslur{1}{m}\qb{0}{m}\enotes
		\bar
		%(7)
		\notes\ibbl{0}{l}{0}\na{l}\zq{l}\isluru{1}{n}\qb{0}{n}\na{k}\zq{k}\qb{0}{m}\zq{j}\qb{0}{l}\tbl{0}\zq{k}\tslur{1}{m}\qb{0}{m}\enotes
		\notes\ibbl{0}{l}{0}\zq{l}\isluru{1}{n}\qb{0}{n}\zq{k}\qb{0}{m}\zq{j}\qb{0}{l}\tbl{0}\zq{l}\tslur{1}{n}\qb{0}{n}\enotes
		\bar
		%(8)
		\NOtes\zq{m}\ql{o}\enotes
		\Notes\ibl{0}{l}{-2}\na{l}\zq{l}\isluru{1}{n}\qb{0}{n}\tbl{0}\na{k}\zq{k}\tslur{1}{m}\qb{0}{m}\enotes
		\bar
		%(9)
		\notes\ibbl{0}{l}{0}\na{l}\zq{l}\isluru{1}{n}\qb{0}{n}\na{k}\zq{k}\qb{0}{m}\zq{j}\qb{0}{l}\tbl{0}\zq{k}\tslur{1}{m}\qb{0}{m}\enotes
		\notes\ibbu{0}{l}{-4}\zcl{l}\ibsluru{1}{n}\qb{0}{n}\qb{0}{l}\enotes
		\notes\raise-2\Interligne\rlap\ds\na{i}\qb{0}{i}\tbu{0}\tbsluru{1}{k}\qb{0}{j}\enotes
	\endextract
	% \endpiece
\musicend{3-45}{第3楽章第45小節から}

\section{第4楽章}
