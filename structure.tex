
\chapter{作品の構造}

\section{概観}

\section{第1楽章}

\musicbegin
	\startbarno=42
	\systemnumbers
	\def\writebarno{\llap{\the\barno\barnoadd}}%
	\def\raisebarno{2\internote}%
	\def\shiftbarno{1.3\Interligne}%
	\def\nbinstruments{1}%   % パート数 2
	\setstaffs{1}{2}%        % 下から1番目は2段
	\setclef{1}{6000}%       % 下から1番目はへ音記号
	\generalsignature{-3}%    % 調号は正の値のときシャープの数
	\generalmeter{\meterfrac{6}{8}}%  % 拍子は8分の6拍子
	\startextract%
		%(1)
		\notes\lpz{C}\cu{C}\ds\ds|\ds\ds\ibsluru{1}{e}\cu{e}\enotes
		\Notes\isluru{0}{c}\qlp{c}|\qu{g}\enotes
		\notes|\tsslur{1}{g}\cl{l}\enotes
		\bar
		%(2)
		\NOTes\tslur{0}{c}\qlp{c}|\isluru{1}{n}\qlp{n}\enotes
		\Notes\sh{c}\qlp{c}|\tslur{1}{n}\ql{n}\enotes
		\Notes|\na{l}\isluru{1}{l}\cl{l}\enotes
		\bar
		%(3)
		\Notes\itieu{0}{d}\qlp{d}|\tslur{1}{n}\ql{n}\ds\enotes
		\Notes\ibl{0}{d}{-3}\ttie{0}\qbp{0}{d}\nbbl{0}\na{c}\isluru{0}{c}\qb{0}{c}\na{b}\qb{0}{b}\tbl{0}\na{a}\qb{0}{a}|\isluru{1}{o}\usf{o}\qlp{o}\enotes
		\bar
		%(4)
		\notes\na{b}\tslur{0}{b}\upz{b}\cl{b}\ds|\tslur{1}{o}\ql{o}\enotes
		\notes\ds|\upz{m}\cl{m}\enotes
		\Notes\qp|\upz{k}\ql{k}\enotes
		\notes\lpz{J}\cu{J}|\upz{j}\cl{j}\enotes
		\bar
		%(5)
		\notes\lpz{G}\cu{G}\ds|\na{i}\upz{i}\cl{i}\ds\enotes
	\zendextract % 頭にzをつけると最後に小節線を表示しない
\musicend{1-42}{第1楽章第42小節から}



\section{第2楽章}

\section{第3楽章: Un poco Allegretto e grazioso}

ブラームスはこの大規模な交響曲の中で, 164小節という小振りな「間奏曲」を用意した.
ベートーヴェン風のスケルツォではなく, より古風なメヌエットのような音楽をここに置いたことは,
ベートーヴェンの交響曲 (例えば第5番) から意識的に距離を置いていることの現れであろう.
しかも, この楽章は全体を通して二拍子で書かれており, 純然たるメヌエットでさえない.
この楽章は完全にブラームス風の音楽であり, この事実ひとつ取ってもブラームスの第1番が「ベートーヴェンの第10番」という評価では言い尽くせないことがよく表れている.

\begin{table}[htbp]
	\centering
	\begin{tabular}{cccc}
		主部 (A) & 中間部 (B) & 再現部 (A') & コーダ \\ \hline
		1--70 & 71--114 & 115--153 & 154--164 \\
		As-Dur, 2/4 & H-Dur, 6/8 & As-Dur, 2/4 & As-Dur, 2/4 (6/8)
	\end{tabular}
	\caption{第3楽章の構成}
	\label{structure of mov3}
\end{table}
構成は比較的単純な三部形式 (A-B-A') だが, 後で見るように再現部A'は主部Aの単調な繰り返しとなることが避けられており,
三部形式の短い楽章にしては変化に富んだ印象を与える.

\musicbegin
	\def\nbinstruments{1}%   % パート数 2
	\setstaffs{1}{2}%        % 下から1番目は2段
	\setclef{1}{6000}%       % 下から1番目はへ音記号
	\generalsignature{-4}%    % 調号は正の値のときシャープの数
	\generalmeter{\meterfrac{2}{4}}%  % 拍子は8分の6拍子
	\startextract%
		%(1)
		\Notes\cmidstaff{\p}\zqu{c}\ibl{0}{H}{0}\qb{0}{HG}|
			\zcharnote{y}{\hspace*{-8truemm}Un poco Allegretto e grazioso}
			\itied{0}{h}\zhl{h}\ibsluru{1}{l}\qu{l}\enotes
		\Notes\zqu{d}\qb{0}{F}\tbl{0}\qb{0}{H}|
			\ibu{1}{k}{-2}\qb{1}{k}\tbsluru{1}{j}\tbu{1}\qb{1}{j}\enotes
		\bar
		%(2)
		\NOtes\ibu{0}{H}{0}\qb{0}{EH}\qb{0}{D}\tbu{0}\qb{0}{H}|
			\zql{e}\ttie{0}\itied{1}{h}\zhl{h}\ibsluru{2}{k}\ibu{1}{k}{-1}\qb{1}{kj}\zql{f}\qb{1}{i}\tbu{1}\tbsluru{2}{j}\qb{1}{j}\enotes
		\bar
		%(3)
		\NOtes\ibu{0}{H}{0}\qb{0}{EH}\qb{0}{D}\tbu{0}\qb{0}{H}|
			\zql{e}\ttie{1}\itied{0}{h}\zhl{h}\ibsluru{2}{k}\ibu{1}{k}{-1}\qb{1}{kj}\zql{f}\qb{1}{i}\tbu{1}\tbsluru{2}{j}\qb{1}{j}\enotes
		\bar
		%(4)
		\Notes\ibu{0}{F}{+1}\qb{0}{C}\tbl{0}\qb{0}{H}|
			\ttie{0}\lq{h}\zhl{e}\ibsluru{1}{i}\ibu{1}{i}{-2}\qb{1}{i}\tbu{1}\tbsluru{1}{h}\qb{1}{h}\enotes
		\Notes\qb{0}{E}\tbu{0}\qb{0}{G}|
			\ibslurd{3}{g}\itied{2}{g}\zql{g}\itieu{0}{i}\qu{i}\enotes
		\bar
		%(5)
		\Notes\ibl{0}{I}{+1}\qb{0}{I}|
			\ttie{0}\zhu{i}\ibl{1}{g}{0}\ttie{2}\qb{1}{g}\enotes
		\Notes\tbl{0}\na{K}\qb{0}{K}|
			\qb{1}{f}\enotes
		\Notes\qb{0}{L}|
			\qb{1}{e}\enotes
		\Notes\tbl{0}\fl{K}\qb{0}{K}|
			\tbslurd{3}{g}\tbl{1}\qb{1}{g}\enotes
	\endextract % 頭にzをつけると最後に小節線を表示しない
\musicend{3-1}{第3楽章冒頭}

第3楽章冒頭はまずチェロのピッチカートに乗ってクラリネットが優雅な旋律を提示する (譜例\ref{3-1}).
ブラームスらしく$5$小節を単位とする変則的な構造を取る. しかも, $2$拍子が$5$小節続くのではなく, $2 + 2 + 3 + 3$という変拍子である.
第6小節からはその反行形が続く.
フルートとファゴットが加わる第11小節からは下降音型を中心とする第2句である (譜例\ref{3-11}).
こちらは冒頭のクラリネット (第1句) と異なり4+4小節の標準的な形である. 第1句と第2句がこの楽章の基本主題を構成する.

\musicbegin
	\startbarno=11
	\systemnumbers
	\def\writebarno{\llap{\the\barno\barnoadd}}%
	\def\raisebarno{2\internote}%
	\def\shiftbarno{1.3\Interligne}%
	\def\nbinstruments{1}%   % パート数
	\setstaffs{1}{1}%        % 下から1番目は2段
	\setclef{1}{0000}%       % 下から1番目はへ音記号
	\generalsignature{-4}%    % 調号は正の値のときシャープの数
	% \generalmeter{\meterfrac{2}{4}}%  % 拍子
	\startextract%
	\NOtes\ibl{0}{l}{-2}\zqp{l}\isluru{1}{n}\qb{0}{n}\enotes
	\notes\tbbl{0}\na{k}\zq{k}\qb{0}{m}\enotes
	\NOtes\zqp{j}\qbp{0}{l}\enotes
	\notes\tbbl{0}\tbl{0}\zq{i}\qb{0}{k}\enotes
	\bar
	\NOtes\ibu{0}{h}{-2}\zqp{h}\qb{0}{j}\enotes
	\notes\tbbu{0}\zq{g}\qb{0}{i}\enotes
	\NOtes\zqp{f}\qbp{0}{h}\enotes
	\notes\tbbu{0}\tbu{0}\zq{e}\tbsluru{1}{l}\qb{0}{g}\enotes
	\bar
	\NOtes\ibl{0}{k}{-2}\na{k}\zqp{k}\isluru{1}{m}\qb{0}{m}\enotes
	\notes\tbbl{0}\zq{j}\qb{0}{l}\enotes
	\NOtes\zqp{i}\qbp{0}{k}\enotes
	\notes\tbbl{0}\tbl{0}\zq{h}\qb{0}{j}\enotes
	\bar
	\NOtes\ibu{0}{h}{-2}\zqp{g}\qb{0}{i}\enotes
	\notes\tbbu{0}\zq{f}\qb{0}{h}\enotes
	\NOtes\qbp{0}{g}\enotes
	\notes\tbbu{0}\tbu{0}\tbsluru{1}{k}\qb{0}{f}\enotes
	\bar
	\NOtes\zhl{g}\ibu{0}{l}{-1}\ibsluru{1}{l}\qb{0}{l}\enotes
	\notes\tbbu{0}\qb{0}{i}\enotes
	\NOtes\qbp{0}{g}\enotes
	\notes\tbbu{0}\tbu{0}\tbsluru{1}{j}\qb{0}{j}\enotes
	\bar
	\NOtes\ibsluru{3}{i}\itieu{2}{i}\lqu{i}\ibl{0}{h}{-1}\ibslurd{1}{h}\qb{0}{h}\enotes
	\notes\tbbl{0}\qb{0}{f}\enotes
	\NOtes\na{d}\zqbp{0}{d}\ibu{1}{i}{+1}\ttie{2}\qbp{1}{i}\enotes
	\notes\tbbl{0}\tbl{0}\tbslurd{1}{h}\zqb{0}{h}\tbbu{1}\tbu{1}\tbsluru{3}{j}\qb{1}{j}\enotes
	\endextract
\musicend{3-11}{第3楽章第11小節から}

第19小節から, やや拡大された形で両旋律が確保される. ここで依然として第1句は9小節単位という変則的な形を,
第2句は4小節単位の標準的な形を保っていることは注目に値する.
また, 拡大部分である第29小節から第31小節にかけて, Vn2にこの曲の基本動機xがさりげなく登場している (譜例\ref{3-29}) ことにも注意したい.


\musicbegin
	\startbarno=29
	\systemnumbers
	\def\writebarno{\llap{\the\barno\barnoadd}}%
	\def\raisebarno{2\internote}%
	\def\shiftbarno{1.3\Interligne}%
	\def\nbinstruments{1}%   % パート数
	\setstaffs{1}{1}%        % 下から1番目は2段
	\setclef{1}{0000}%       % 下から1番目はへ音記号
	\generalsignature{-4}%    % 調号は正の値のときシャープの数
	% \generalmeter{\meterfrac{2}{4}}%  % 拍子
	\startextract%
	\Notes\isluru{1}{i}\itieu{2}{i}\ql{i}\enotes
	\bar
	\Notes\ttie{2}\ql{i}\na{i}\ql{i}\enotes
	\bar
	\Notes\ql{j}\ql{k}\enotes
	\bar
	\notes\ibl{0}{l}{-2}\na{k}\qb{0}{k}\curve 543\tslur{1}{j}\qb{0}{j}\isluru{1}{k}\upz{i}\qb{0}{i}\tbl{0}\tslur{1}{j}\upz{h}\qb{0}{h}\enotes
	\endextract
\musicend{3-29}{第3楽章第29小節2拍目からのVn2}

第45小節でヘ短調に落ち込むと, クラリネット, 次いでフルートとオーボエに新しいリズムが出る (譜例\ref{3-45}) が,
これは前半は第2句, 後半は第1句に基づく経過句である.

\begin{music}
	\startbarno=45
	\systemnumbers
	\def\writebarno{\llap{\the\barno\barnoadd}}%
	\def\raisebarno{2\internote}%
	\def\shiftbarno{1.3\Interligne}%
	\def\nbinstruments{1}%   % パート数
	\setstaffs{1}{1}%        % 下から1番目は2段
	\setclef{1}{0000}%       % 下から1番目はへ音記号
	\generalsignature{-4}%    % 調号は正の値のときシャープの数
	% \generalmeter{\meterfrac{2}{4}}%  % 拍子
	\startextract%
	% \startpiece
		%(1)
		\Notes\ds\isluru{1}{j}\cl{j}\enotes
		\bar
		%(2)
		\Notes\ibu{0}{i}{-2}\na{i}\qb{0}{i}\qb{0}{h}\qb{0}{g}\tbu{0}\curve 522\tbsluru{1}{l}\qb{0}{f}\enotes
		\bar
		%(3)
		\Notes\ibl{0}{i}{+2}\na{i}\isluru{1}{i}\qbp{0}{i}\enotes
		\notes\tbbl{0}\tbl{0}\tslur{1}{k}\qb{0}{k}\enotes
		\notes\ibbl{0}{j}{+2}\isluru{2}{j}\qb{0}{jon}\tbbl{0}\tbl{0}\curve512\tslur{2}{m}\isluru{1}{m}\qb{0}{m}\enotes
		\bar
		%(4)
		\Notes\ibl{0}{l}{-2}\na{l}\qb{0}{l}\qb{0}{k}\qb{0}{j}\tbl{0}\tbsluru{1}{i}\fl{i}\qb{0}{i}\enotes
		\bar
		%(5)
		\NOtes\ibl{0}{l}{+2}\fl{l}\isluru{1}{l}\qbp{0}{l}\enotes
		\notes\tbbl{0}\tbl{0}\tslur{1}{n}\fl{n}\qb{0}{n}\enotes
		\notes\ibbl{0}{j}{+2}\isluru{2}{m}\qb{0}{mrq}\tbbl{0}\tbl{0}\curve511\tslur{2}{p}\qb{0}{p}\enotes
		\bar
		% \alaligne
		%(6)
		\Notes\zq{m}\ql{o}\enotes
		\Notes\ibl{0}{l}{-2}\na{l}\zq{l}\isluru{1}{n}\qb{0}{n}\tbl{0}\na{k}\zq{k}\tslur{1}{m}\qb{0}{m}\enotes
		\bar
		%(7)
		\notes\ibbl{0}{l}{0}\na{l}\zq{l}\isluru{1}{n}\qb{0}{n}\na{k}\zq{k}\qb{0}{m}\zq{j}\qb{0}{l}\tbl{0}\zq{k}\tslur{1}{m}\qb{0}{m}\enotes
		\notes\ibbl{0}{l}{0}\zq{l}\isluru{1}{n}\qb{0}{n}\zq{k}\qb{0}{m}\zq{j}\qb{0}{l}\tbl{0}\zq{l}\tslur{1}{n}\qb{0}{n}\enotes
		% \bar
		% %(8)
		% \NOtes\zq{m}\ql{o}\enotes
		% \Notes\ibl{0}{l}{-2}\na{l}\zq{l}\isluru{1}{n}\qb{0}{n}\tbl{0}\na{k}\zq{k}\tslur{1}{m}\qb{0}{m}\enotes
		% \bar
		% %(9)
		% \notes\ibbl{0}{l}{0}\na{l}\zq{l}\isluru{1}{n}\qb{0}{n}\na{k}\zq{k}\qb{0}{m}\zq{j}\qb{0}{l}\tbl{0}\zq{k}\tslur{1}{m}\qb{0}{m}\enotes
		% \notes\ibbu{0}{l}{-4}\zcl{l}\ibsluru{1}{n}\qb{0}{n}\qb{0}{l}\enotes
		% \notes\raise-2\Interligne\rlap\ds\na{i}\qb{0}{i}\tbu{0}\tbsluru{1}{k}\qb{0}{j}\enotes
	\endextract
	% \endpiece
\musicend{3-45}{第3楽章第45小節2拍目から}

第62小節で変イ長調に戻ると第1句を再現するが, これはあっさりと流して遠隔調であるロ長調の中間部へと続く.
ここで第65小節からの木管楽器の動きが第4楽章の第58小節や第295小節を思い出させる, と言うと穿ちすぎだろうか.
その解釈に立ってこの箇所を第3楽章第28小節からの木管および譜例\ref{3-29}に関する上の記述と比較すると,
主部Aにおいて3回演奏されるこの主要主題は, 最初 (第1小節から) は含みのない形で提示されるが,
2回目 (第19小節から) は第1楽章に, 3回目 (第62小節から) は第4楽章に寄せている, ということになる.
中間部Bが第1楽章の追憶に捧げられ, 再現部A'が第4楽章の準備段階となっていることを踏まえると, この見方は如何にもありそうに思える.

\musicbegin
	\shosetu{71}
	\def\nbinstruments{1}%   % パート数
	% \setstaffs{1}{2}%        % 下から1番目は2段
	% \setclef{1}{6000}%       % 下から1番目はへ音記号
	\generalsignature{+5}%    % 調号は正の値のときシャープの数
	\generalmeter{\meterfrac{6}{8}}%  % 拍子
	\startextract%
		\notes\ds\ibu{0}{k}{0}\islurd{1}{k}\qb{0}{k}\tbu{0}\qb{0}{k}\enotes
		\notes\tslur{1}{k}\zqup{k}\raise-3\Interligne\ds\ibl{1}{f}{-2}\isluru{2}{f}\qb{1}{f}\tbl{1}\qb{1}{d}\enotes
		\bar
		\notes\tslur{2}{b}\zql{b}\ds\ibu{0}{k}{0}\islurd{1}{k}\qb{0}{k}\tbu{0}\zqb{0}{k}\raise-3\Interligne\ds\enotes
		\notes\tslur{1}{k}\zqup{k}\raise-3\Interligne\ds\ibl{1}{d}{-2}\isluru{2}{d}\qb{1}{d}\tbl{1}\qb{1}{b}\enotes
		\bar
		\notes\tslur{2}{N}\zqlp{N}\ds\ibl{0}{i}{+1}\zq{i}\isluru{1}{k}\qb{0}{k}\tbl{0}\zq{j}\qb{0}{l}\enotes
		\notes\ibl{0}{k}{-2}\zq{k}\qb0{m}\zq{j}\qb0{l}\tbl0\zq{i}\qb0{k}\enotes
		\bar
		\notes\ibu{0}{h}{0}\zq{h}\qb0{j}\zq{g}\qb0{i}\tbu0\zq{h}\qb0{j}\enotes
		\notes\ibl{0}{i}{+2}\zq{i}\qb0{k}\zq{j}\qb0{l}\tbl0\zq{k}\tslur{1}{m}\qb0{m}\enotes
		\bar
		\Notes\zq{h}\qu{j}\ds\enotes
	\zendextract % 頭にzをつけると最後に小節線を表示しない
\musicend{3-71}{第3楽章第71小節から}

中間部は八分の六拍子でのDis音の連打から始まる (譜例\ref{3-71}). これは直前の第65小節から第70小節にかけて何度も強調されるB音から誘導されたものであるが,
第83小節でホルンとトランペットが強い調子でFis音を連打するに至って, これが第1楽章のオルゲルクンプトの回想であることが明らかになる (第1楽章展開部の金管楽器の用法を思い出そう).
ただ, 第73小節からの順次進行は節回しや3度を好む傾向という点ではむしろ第2楽章の主要主題を思わせる.

続くトリオ風の音楽 (第87小節から) は不安定な和音進行が特徴的である.
それまではロ調まわりで安定していたが, ここで基本動機xがバス声部に潜ることによって多彩な和声が導き出される.

% 和声進行もうちょっと詳しく書く

第107小節でDis音の連打だけが残ると, これをEs音に読み替えることで変イ短調でEs-Des-Ces-Bの順次下降が弦楽器のユニゾンで奏される (譜例\ref{3-108}).
これはもちろん中間部Bを打ち切り主部Aの再現を開始するという合図であるが, 同時に第4楽章冒頭の予告にもなっている (譜例\ref{4-1}).

\musicbegin
	\shosetu{108}
	\def\nbinstruments{1}%   % パート数
	\setstaffs{1}{1}%        % 下から1番目は2段
	\setclef{1}{0000}%       % 下から1番目はへ音記号
	\generalsignature{-4}%    % 調号は正の値のときシャープの数
	\generalmeter{\meterfrac{2}{4}}%  % 拍子
	\startextract%
	\NOtes\isluru{1}{s}\ql{s}\enotes
	\Notes\ibl{0}{p}{-2}\qb{0}{r}\tbl{0}\fl{q}\qb{0}{q}\enotes
	\bar
	\Notes\tslur{1}{p}\cl{p}\ds\enotes
	\NOtes\qp\enotes
	\bar
	\NOtes\qu{e}\qp\enotes
	\bar
	\NOtes\qu{d}\fl{c}\qu{c}\enotes
	\bar
	\Notes\cu{b}\ds\enotes
	\zendextract
\musicend{3-108}{第3楽章第108小節からのVn1}

第115小節からの再現部A'ではまずクラリネットが冒頭主題を再現するが,
フルートとオーボエによる中間部Bに基づく対旋律が付加されているために雰囲気は冒頭とはやや異なる.
次いで主部では旋律\ref{3-1}の反行形が提示された箇所は, ヴァイオリンの目新しいがどこか懐かしい旋律に置き換えられる (譜例\ref{3-118}).
これは暗に第4楽章第1主題 (譜例\ref{4-61})を指し示しているが, 両者の類似は単なる旋律上のものだけではない.
中間部の和声, リズムともに複雑な領域を抜けた後で提示されるこの変ニ長調の控えめな旋律 (molto dolceの指定付き) によって,
聞き手の緊張が和らげられ, 自然にリラックスした状態に落ち着くことになる.
この効果はまさに全曲の中で第4楽章第1主題が果たす役割と同一のものである.

\musicbegin
	\shosetu{118}
	\def\nbinstruments{1}%   % パート数
	\setstaffs{1}{1}%        % 下から1番目は2段
	\setclef{1}{0000}%       % 下から1番目はへ音記号
	\generalsignature{-4}%    % 調号は正の値のときシャープの数
	% \generalmeter{\meterfrac{2}{4}}%  % 拍子
	\startextract%
	\NOtes\isluru{1}{i}\itieu{2}{i}\ql{i}\enotes
	\bar
	\NOtes\ttie{2}\ql{i}\tslur{1}{j}\ql{j}\enotes
	\bar
	\notesp\ibl{0}{k}{0}\isluru{1}{k}\qb{0}{k}\enotes
	\notes\nbbl{0}\qb{0}{l}\tbbl{0}\qb{0}{m}\enotes
	\notesp\qb{0}{l}\tbl{0}\qb{0}{k}\enotes
	\bar
	\notesp\ibl{0}{j}{0}\qb{0}{j}\qb{0}{l}\qb{0}{k}\tbl{0}\qb{0}{j}\enotes
	\bar
	\notesp\ibl{0}{i}{-1}\qb{0}{i}\enotes
	\notes\nbbl{0}\qb{0}{j}\tbbl{0}\qb{0}{k}\enotes
	\notesp\qb{0}{j}\tbl{0}\qb{0}{g}\enotes
	\bar
	\notes\ibu{0}{h}{0}\nbbu{0}\qb0{h}\tbbu{0}\qb0{i}\enotes
	\nnotes\triolet{b}\nbbu{0}\qb0{h}\qb0{g}\tbbu{0}\tbu{0}\qb0{h}\enotes
	\notesp\ibl{0}{j}{-2}\qb0{j}\tbl{0}\curve 542\tslur{1}{i}\qb0{i}\enotes
	% \bar
	% \Notes\qu{g}\enotes
	% \notesp\ibu{0}{f}{-2}\islurd{1}{f}\qb0{f}\tbu{0}\na{d}\tslur{1}{d}\qb{0}{d}\enotes
	% \bar
	% \NOtes\hu{e}\enotes
	\endextract
\musicend{3-118}{第3楽章第118小節2拍目からのVn1}

第126小節からの第2句は後半が大幅に拡大され, 主部にあったヘ短調の経過句を飲み込んで, 第144小節での第1句の再提示へと繋がる.
一瞬だけ基本主題xを思い出す (第148小節) が, しかしその思いを振り切って変ニ短調に傾斜したコーダへ入る.

コーダは第154小節から第164小節というごく短いもので, 中間部Bの追憶となっている.
ここではブラームスが好んで使用した二連符と三連符の交錯がくすんだニュアンスという絶妙な効果を発揮している.
また, 下降音型は短調と, 上昇音型は長調と結び付けられており,
主和音の明確な提示を回避しながらも十分な音楽的効果が発揮される. この手法は後の交響曲第3番, 特に第4楽章第2主題を思わせる.

第3楽章は, 大規模な第1楽章, 第4楽章の間にあって規模が小さすぎ, 音楽的にも内容に乏しいとの批判がある.
% リファレンスが必要
しかしこの批判は妥当ではない.
上で見たように, さりげない形で\footnote{この点でもベートーヴェンの第5番あるいは第9番との相違は際立っている.}第1楽章の内容を再提示し,
フィナーレへの足掛かりを用意するという点にこの楽章の意義がある.
この観点からするとブラームスが書き下したこの音楽は必要な内容を完全に含んでおり, 作品全体を傑作たらしめるのに十分であると言えよう.


\section{第4楽章}
