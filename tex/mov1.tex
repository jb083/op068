
\section{第1楽章: Un poco sostenuto-Allegro}

% 最初に軽く概観

この楽章は大規模な序奏を持つソナタ形式で, 全体の構成を表\ref{structure of mov1}に示す
(便宜上展開部を\ind{D}{1}, \ind{D}{2}, \ind{D}{3}に分けた).
この表では再現部の入りを第343小節としたが, この点については後で詳細に議論する.
\begin{table}[htbp]
	\centering
	\begin{tabular}{c|ccc|ccc|ccc|c}
		序奏I & \multicolumn{3}{c|}{提示部E} & \multicolumn{3}{c|}{展開部D} &
			\multicolumn{3}{c|}{再現部R} & コーダC \\ \hline
		1--37 & \multicolumn{3}{c|}{38--188} & \multicolumn{3}{c|}{189--342} &
			\multicolumn{3}{c|}{343--462} & 463--511 \\
		& \ind{E}{1} & \ind{E}{2} & \ind{E}{c} & \ind{D}{1} & \ind{D}{2} & \ind{D}{3} &
			\ind{R}{1} & \ind{R}{2} & \ind{R}{c} & \\ \cline{2-10}
		& 38- & 130- & 159- & 189- & 225- & 293- & 343- & 403- & 430- & \\
		c & c & Es & es & H--h--c & Ges--c & c--fis & c & C & c & c--C
	\end{tabular}
	\caption{第1楽章の構成}
	\label{structure of mov1}
\end{table}

% 序奏はティンパニ, コントラバス, コントラファゴットのC音連打の上に何重にも積み重なった上昇音型と下降音型によって開始される.
% これらの要素はいずれもこの作品全体を支配する基本的なもので, 本稿ではこれを順に基本動機Z, X, Yと呼称することにする.

\musicbegin
	\shosetu{42}
	\def\nbinstruments{1}%   % パート数 2
	\setstaffs{1}{2}%        % 下から1番目は2段
	\setclef{1}{6000}%       % 下から1番目はへ音記号
	\generalsignature{-3}%    % 調号は正の値のときシャープの数
	\generalmeter{\meterfrac{6}{8}}%  % 拍子は8分の6拍子
	\startextract%
		%(1)
		\notes\lpz{C}\cu{C}\ds\ds|\ds\ds\ibsluru{1}{e}\cu{e}\enotes
		\Notes\isluru{0}{c}\qlp{c}|\qu{g}\enotes
		\notes|\tsslur{1}{g}\cl{l}\enotes
		\bar
		%(2)
		\NOTes\tslur{0}{c}\qlp{c}|\isluru{1}{n}\qlp{n}\enotes
		\Notes\sh{c}\qlp{c}|\tslur{1}{n}\ql{n}\enotes
		\Notes|\na{l}\isluru{1}{l}\cl{l}\enotes
		\bar
		%(3)
		\Notes\itieu{0}{d}\qlp{d}|\tslur{1}{n}\ql{n}\ds\enotes
		\Notes\ibl{0}{d}{-3}\ttie{0}\qbp{0}{d}|\isluru{1}{o}\usf{o}\qlp{o}\enotes
		\notes\nbbl{0}\na{c}\isluru{0}{c}\qb{0}{c}\na{b}\qb{0}{b}\tbl{0}\na{a}\qb{0}{a}|\enotes
		\bar
		%(4)
		\notes\na{b}\tslur{0}{b}\upz{b}\cl{b}\ds|\tslur{1}{o}\ql{o}\enotes
		\notes\ds|\upz{m}\cl{m}\enotes
		\Notes\qp|\upz{k}\ql{k}\enotes
		\notes\lpz{J}\cu{J}|\upz{j}\cl{j}\enotes
		\bar
		%(5)
		\notes\lpz{G}\cu{G}\ds|\na{i}\upz{i}\cl{i}\ds\enotes
	\zendextract % 頭にzをつけると最後に小節線を表示しない
\musicend{1-42}{第1楽章第42小節から}

% 最後に序奏とコーダのテンポ設定の問題に触れる
