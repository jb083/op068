
\section{第1楽章: Un poco sostenuto-Allegro}

% 最初に軽く概観

この楽章は大規模な序奏を持つソナタ形式で, 全体の構成を表\ref{structure of mov1}に示す
(便宜上展開部を\ind{D}{1}, \ind{D}{2}, \ind{D}{3}に分けた).
この表では再現部の入りを第343小節としたが, この点については後で詳細に議論する.
\begin{table}[htbp]
	\centering
	\begin{tabular}{c|ccc|ccc|ccc|c}
		序奏I & \multicolumn{3}{c|}{提示部E} & \multicolumn{3}{c|}{展開部D} &
			\multicolumn{3}{c|}{再現部R} & コーダC \\ \hline
		1--37 & \multicolumn{3}{c|}{38--188} & \multicolumn{3}{c|}{189--342} &
			\multicolumn{3}{c|}{343--462} & 463--511 \\
		& \ind{E}{1} & \ind{E}{2} & \ind{E}{c} & \ind{D}{1} & \ind{D}{2} & \ind{D}{3} &
			\ind{R}{1} & \ind{R}{2} & \ind{R}{c} & \\ \cline{2-10}
		& 38- & 130- & 159- & 189- & 225- & 293- & 343- & 403- & 430- & \\
		c & c & Es & es & H--h--c & Ges--c & c--fis & c & C & c & c--C
	\end{tabular}
	\caption{第1楽章の構成}
	\label{structure of mov1}
\end{table}


\clearpage
\musicbegin
	\setmaxinstruments{20}
	\setmaxgroups{8}
	\setmaxslurs{99}
	\def\nbinstruments{14}%   % パート数
	\akkoladen{{1}{5}{7}{9}{10}{14}}
	% \curlybrackets{1245}
	% \sepbarrules % 一旦すべての小節線を切り離す
	% set clefs and names
	\setclef{14}{0}\setname{14}{Fl\hspace{8truemm}}%
	\setclef{13}{0}\setname{13}{Ob\hspace{8truemm}}%
	\setclef{12}{0}\setname{12}{Kl (B)\hspace{10truemm}}%
	\setclef{11}{6}\setname{11}{Fg\hspace{8truemm}}%
	\setclef{10}{6}\setname{10}{Kfg\hspace{8truemm}}%
	\setclef{9}{0}\setname{9}{Hrn (C)\hspace{12truemm}}%
	\setclef{8}{0}\setname{8}{Hrn (Es)\hspace{12truemm}}%
	\setclef{7}{0}\setname{7}{Tr (C)\hspace{10truemm}}%
	\setclef{6}{6}\setname{6}{Pk\hspace{8truemm}}%
	\setclef{5}{0}\setname{5}{Vn1\hspace{7truemm}}%
	\setclef{4}{0}\setname{4}{Vn2\hspace{8truemm}}%
	\setclef{3}{3}\setname{3}{Va\hspace{8truemm}}%
	\setclef{2}{6}\setname{2}{Vc\hspace{8truemm}}%
	\setclef{1}{6}\setname{1}{Kb\hspace{8truemm}}%
	% set chord
	\generalsignature{-3}%    % 調号は正の値のときシャープの数
	\setsign{12}{-1}
	\setsign{9}{0}
	\setsign{8}{0}
	\setsign{7}{0}
	\setsign{6}{0}
	% set metre
	\generalmeter{\meterfrac{6}{8}}%
	\startextract%
		%(1)
		\setclef{2}{0}\zchangeclefs
		\notes
			\ibu{0}{J}{0}\qb{0}{JJ}\tbu{0}\qb{0}{J}\ibu{0}{J}{0}\qb{0}{JJ}\tbu{0}\qb{0}{J}&
			\islurd{2}{c}\itied{1}{c}\qup{c**}\ttie{1}\qu{c*}\sh{c}\itied{1}{c}\cu{c}&
			\pt{c}\zq{c}\isluru{3}{j}\qlp{j**}\pt{g}\zq{g}\qlp{i**}&
			\itieu{5}{j}\qlp{j**}\ttie{5}\ql{j*}\sh{j}\itieu{5}{j}\cl{j}&%Vn2
			&%\isluru{4}{q}\itieu{3}{q}\qlp{q**}\ttie{3}\ql{q*}\sh{q}\itied{3}{q}\cl{q}&
			\ibu{0}{J}{0}\qb{0}{JJ}\tbu{0}\qb{0}{J}\ibu{0}{J}{0}\qb{0}{JJ}\tbu{0}\qb{0}{J}&
			\pt{c}\itied{8}{c}\zq{c}\itieu{9}{j}\qup{j**}\ttie{8}\zq{c}\ttie{9}\cu{j}\ds\ds&%Tp
			\qp\cu{*}\ds\pt{'e}\zq{e}\isluru{19}{g}\qlp{g**}&
			&
			\ibu{0}{C}{0}\qb{0}{CC}\tbu{0}\qb{0}{C}\ibu{0}{C}{0}\qb{0}{CC}\tbu{0}\qb{0}{C}&
			\pt{J}\zq{J}\isluru{21}{c}\qlp{c**}\pt{N}\zq{N}\qlp{b**}&%Fg
			\pt{k}\zq{k}\isluru{22}{r}\qlp{r**}\pt{'h}\zq{h}\qlp{j**}&
			\pt{j}\zq{j}\isluru{23}{q}\qlp{q**}\pt{'g}\zq{g}\qlp{i**}&
			\zcharnote{v}{\hspace*{-8truemm}Un poco sostenuto}\pt{j}\zq{j}\isluru{24}{q}\qlp{q**}\pt{'g}\zq{g}\qlp{i**}
		\enotes
		\bar
		%(2)
		\notes
			\trmu{J}\zhup{J}&
			\ttie{1}\qu{c*}\itied{1}{d}\cu{d}&
			\sh{f}\na{h}\pt{f}\zq{f}\qlp{h**}&
			\ttie{5}\ql{j*}\itieu{5}{k}\cl{k}&%Vn2
			&%\ttie{3}\ql{q*}\itied{3}{r}\cl{r}&
			\trmu{J}\zhup{J}&
			&%Tp
			\pt{'e}\lq{_e}\qlp{^f}&
			&
			\trmu{C}\zhup{C}&
			\sh{M}\na{a}\pt{M}\zq{M}\qlp{a**}&%Fg
			\sh{'g}\na{i}\pt{g}\zq{g}\qlp{i**}&
			\sh{'f}\na{h}\pt{f}\zq{f}\qlp{h**}&
			\sh{'f}\na{h}\pt{f}\zq{f}\qlp{h**}
		\enotes
		\nnotes
			&
			\ibu{0}{d}{+3}\ttie{1}\qbp{0}{d*}\nbbu{0}\qb{0}{e}\qb{0}{f}\tbbu{0}\tbu{0}\tslur{2}{g}\itied{1}{g}\qb{0}{g}&
			\na{f}\fl{h}\zql{f}\ql{h**}\zq{e}\tslur{3}{g}\cl{g}&
			\ibl{5}{k}{+3}\ttie{5}\qbp{5}{k*}\nbbl{5}\qb{5}{l}\qb{5}{m}\tbbl{5}\tbl{5}\itieu{5}{n}\qb{5}{n}&%Vn2
			&%\ibl{0}{r}{+3}\ttie{3}\qbp{0}{r*}\nbbl{0}\qb{0}{s}\qb{0}{t}\tbbl{0}\tbl{0}\tslur{4}{u}\itied{3}{u}\qb{0}{u}
			&
			&%Tp
			\na{'f}\zq{d}\ql{f**}\zq{c}\cl{e}&
			&
			&
			\na{M}\fl{a}\zql{M}\ql{a**}\zq{L}\tslur{21}{N}\cl{N}&%Fg
			\na{'g}\fl{i}\zql{g}\ql{i**}\zq{f}\tslur{22}{h}\cl{h}&
			\na{'f}\fl{h}\zql{f}\ql{h**}\zq{e}\tslur{23}{g}\cl{g}&
			\na{'f}\fl{h}\zql{f}\ql{h**}\zq{e}\tslur{24}{g}\cl{g}
		\enotes
		\def\atnextbar{\znotes&&&&&&\centerbar{\cpause}\en}%
		\bar
		%(3)
		\notes
			\trmu{J}\zhup{J}&
			\ttie{1}\islurd{2}{g}\qu{g*}\itied{1}{h}\cu{h}\ttie{1}\qu{h*}\na{h}\itied{1}{h}\cu{h}&
			\pt{d}\zq{d}\isluru{3}{f}\qlp{f**}\pt{c}\zq{c}\qlp{e**}&
			\ttie{5}\ql{n*}\itieu{5}{o}\cl{o}\ttie{5}\ql{o*}\na{o}\itieu{5}{o}\cl{o}&%Vn2
			&%\ttie{3}\isluru{4}{u}\ql{u*}\itieu{3}{v}\cl{v}\ttie{3}\ql{v*}\na{v}\itieu{3}{v}\ql{v}&
			\trmu{J}\zhup{J}&
			&%Tp
			\pt{i}\zq{i}\qlp{k**}\pt{c}\zq{c}\qup{j**}&
			&
			\trmu{C}\zhup{C}&
			\pt{K}\zq{K}\isluru{21}{M}\qlp{M**}\pt{J}\zq{J}\tslur{21}{L}\qlp{L**}&%Fg
			\pt{'e}\zq{e}\isluru{22}{g}\qlp{g**}\pt{d}\zq{d}\tslur{22}{f}\qlp{f**}&
			\pt{'d}\zq{d}\isluru{23}{f}\qlp{f**}\pt{c}\zq{c}\tslur{23}{e}\qlp{e**}&
			\pt{'d}\zq{d}\isluru{24}{f}\qlp{f**}\pt{'c}\zq{c}\tslur{24}{e}\qlp{e**}
		\enotes
		\def\atnextbar{\znotes&&&&&&\centerbar{\cpause}\en}%
		\bar
		%(4)
		\notes
			\trmu{J}\zhup{J}&
			\ttie{1}\qu{h*}\itied{1}{i}\cu{i}\ttie{1}\qu{i*}\fl{h}\tslur{2}{h}\itied{1}{h}\qu{h}&
			\pt{b}\zq{b}\qlp{d**}\ibl{3}{c}{+2}\zqb{3}{b}\qb{3}{d}\zqb{3}{c}\qb{3}{e}\tbl{3}\zqb{3}{d}\tslur{3}{f}\qb{3}{f}&
			\ttie{5}\ql{o*}\itieu{5}{p}\cl{p}\ttie{5}\ql{p*}\fl{o}\itieu{5}{o}\cl{o}&%Vn2
			&%\ttie{3}\ql{v*}\itieu{3}{w}\cl{w}\ttie{3}\ql{w*}\fl{v}\tslur{4}{v}\itied{3}{v}\ql{v}&
			\trmu{J}\zhup{J}&
			&%Tp
			\zq{g}\curve 234\tslur{19}{i}\qu{i*}\ds\qp\cu{*}\ds&
			&
			\trmu{C}\zhup{C}&
			\pt{I}\zq{I}\isluru{21}{K}\qlp{K**}\ibl{3}{J}{+2}\zqb{3}{I}\qb{3}{K}\zqb{3}{J}\qb{3}{L}\tbl{3}\zqb{3}{K}\tslur{21}{M}\qb{3}{M}&%Fg
			\pt{j}\zq{j}\isluru{22}{l}\qlp{l**}\ibu{3}{e}{+2}\zqb{3}{c}\qb{3}{e}\zqb{3}{d}\qb{3}{f}\tbu{3}\zqb{3}{e}\tbsluru{22}{l}\qb{3}{g}&
			\pt{'b}\zq{b}\isluru{23}{d}\qlp{d**}\ibl{3}{c}{+2}\zqb{3}{b}\qb{3}{d}\zqb{3}{c}\qb{3}{e}\tbl{3}\zqb{3}{d}\tslur{23}{f}\qb{3}{f}&
			\pt{''b}\zq{b}\isluru{24}{d}\qlp{d**}\ibl{3}{b}{+2}\zqb{3}{b}\qb{3}{d}\zqb{3}{c}\qb{3}{e}\tbl{3}\zqb{3}{d}\tslur{24}{f}\qb{3}{f}
		\enotes
		\def\atnextbar{\znotes&&&&&&\centerbar{\cpause}\en}%
		\bar
		%(5)
		\nnotes
			\trmu{J}\zhup{J}&
			\ibu{0}{g}{0}\ttie{1}\ibsluru{2}{h}\qbp{0}{h*}\nbbu{0}\qb{0}{g}\qb{0}{N}\tbbu{0}\tbu{0}\qb{0}{d}&
			\na{b}\pt{b}\zq{b}\isluru{3}{d}\qlp{d**}&
			\ibl{5}{h}{0}\ttie{5}\qbp{5}{o*}\nbbl{5}\qb{5}{n}\qb{5}{g}\tbbl{5}\tbl{5}\qb{5}{k}&%Vn2
			&
			\trmu{J}\zhup{J}&
			&%Tp
			&
			&
			\trmu{C}\zhup{C}&
			\na{I}\pt{I}\zq{I}\isluru{21}{K}\qlp{K**}&%Fg
			\sh{c}\pt{c}\zq{c}\isluru{22}{e}\qlp{e**}&
			\na{'b}\pt{b}\zq{b}\isluru{23}{d}\qlp{d**}&
			\na{''b}\pt{b}\zq{b}\isluru{24}{d}\qlp{d**}
		\enotes
		\notes
			&
			\qu{g*}\tbsluru{2}{f}\cu{f}&
			\ibl{3}{c}{+2}\zqb{3}{b}\qb{3}{d}\zqb{3}{c}\qb{3}{e}\tbl{3}\zqb{3}{d}\tslur{3}{f}\qb{3}{f}&
			\ql{n*}\cl{m}&%Vn2
			&
			&
			&%Tp
			&
			&
			&
			\ibl{3}{J}{+2}\zqb{3}{I}\qb{3}{K}\zqb{3}{J}\qb{3}{L}\tbl{3}\zqb{3}{K}\tslur{21}{M}\qb{3}{M}&%Fg
			\ibl{3}{d}{+2}\zqb{3}{c}\qb{3}{e}\zqb{3}{d}\qb{3}{f}\tbl{3}\zqb{3}{e}\tslur{22}{g}\qb{3}{g}&
			\ibl{3}{'c}{+2}\zqb{3}{b}\qb{3}{d}\zqb{3}{c}\qb{3}{e}\tbl{3}\zqb{3}{d}\tslur{23}{f}\qb{3}{f}&
			\ibl{3}{''c}{+2}\zqb{3}{b}\qb{3}{d}\zqb{3}{c}\qb{3}{e}\tbl{3}\zqb{3}{d}\tslur{24}{f}\qb{3}{f}
		\enotes
		\def\atnextbar{\znotes&&&&&&\centerbar{\cpause}&\centerbar{\cpause}\en}%
	\endextract % 頭にzをつけると最後に小節線を表示しない
\musicend{1-1}{第1楽章冒頭}
\newpage


序奏はティンパニ, コントラバス, コントラファゴットのC音連打の上に何重にも積み重なった上昇音型と下降音型によって開始される.
これらの要素はいずれもこの作品全体を支配する基本的なもので, 本稿ではこれを順に基本動機Z, X, Yと呼称することにする.

\musicbegin
	\shosetu{42}
	\def\nbinstruments{1}%   % パート数 2
	\setstaffs{1}{2}%        % 下から1番目は2段
	\setclef{1}{6000}%       % 下から1番目はへ音記号
	\generalsignature{-3}%    % 調号は正の値のときシャープの数
	\generalmeter{\meterfrac{6}{8}}%  % 拍子は8分の6拍子
	\startextract%
		%(1)
		\notes\zcharnote{a}{\hspace*{-1truemm}{\footnotesize (Vc)}}\lpz{C}\cu{C}\ds\ds|\ds\ds\zcharnote{p}{\hspace*{-2truemm}{\footnotesize (Vn1)}}\ibsluru{1}{e}\cu{e}\enotes
		\Notes\isluru{0}{c}\qlp{c}|\qu{g}\enotes
		\notes|\tsslur{1}{g}\cl{l}\enotes
		\bar
		%(2)
		\NOTes\tslur{0}{c}\qlp{c}|\isluru{1}{n}\qlp{n}\enotes
		\Notes\sh{c}\qlp{c}|\tslur{1}{n}\ql{n}\enotes
		\Notes|\na{l}\isluru{1}{l}\cl{l}\enotes
		\bar
		%(3)
		\Notes\itieu{0}{d}\qlp{d}|\tslur{1}{n}\ql{n}\ds\enotes
		\Notes\ibl{0}{d}{-3}\ttie{0}\qbp{0}{d}|\isluru{1}{o}\usf{o}\qlp{o}\enotes
		\notes\nbbl{0}\na{c}\isluru{0}{c}\qb{0}{c}\na{b}\qb{0}{b}\tbl{0}\na{a}\qb{0}{a}|\enotes
		\bar
		%(4)
		\notes\na{b}\tslur{0}{b}\upz{b}\cl{b}\ds|\tslur{1}{o}\ql{o}\enotes
		\notes\ds|\upz{m}\cl{m}\enotes
		\Notes\qp|\upz{k}\ql{k}\enotes
		\notes\lpz{J}\cu{J}|\upz{j}\cl{j}\enotes
		\bar
		%(5)
		\notes\lpz{G}\cu{G}\ds|\na{i}\upz{i}\cl{i}\ds\enotes
	\zendextract % 頭にzをつけると最後に小節線を表示しない
\musicend{1-42}{第1楽章第42小節から}

% 最後に序奏とコーダのテンポ設定の問題に触れる
